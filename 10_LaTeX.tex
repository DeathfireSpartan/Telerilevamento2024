% In LaTeX le funzioni iniziano con \ e sono contenute all'interno di parentesi graffe {}
% Per i commenti

% \documentclass{nometemplate} --> Indica il tipo di documento che andremo a creare --> template di base
% \usepackage{nomepacchetto} --> Una funzione per aggiungere pacchetti aggiuntivi a LaTeX 

% \title{Il mio primo documento in LaTeX} --> Titolo del documento
% \author{Luca Telloli} --> Autore del documento
% \date{07 Maggio 2024} --> Imposto la data 
% se \date è mancante Overleaf va a prendere la data dal computer

% \begin{document} e \end{document} --> Da usare in tandem per iniziare e finire il documento
% \maketitle --> Copia tutta la parte iniziale del documento, cioè l'Header
% \section{Introduction} --> Funge da inizio paragrafo e numera il titolo
% \section*{Introduction} --> L'asterisco NON fa mettere il numero
% Le righe vuote tra i paragrafi mettono a capo il testo con un'indentazione
% \noindent --> Rimuove l'identazione --> fa schifo
% \documentclass[12pt]{article} --> [12pt] setta la dimensione del testo a 12 per TUTTO il documento
% \tableofcontents --> Crea l'indice del documento

% \usepackage{hyperref} --> Funzione che aggiunge degli hyperlink nell'indice che reindirizzano ai vari capitoli

% \begin{equation} e \end{equation} --> Permette di creare un'equazione 
% PM = \sqrt[3]{\frac{Meloni_1 \times Fame_2}{Prosciutto^3}} --> La funzione \frac crea un funzione del tipo a/b 
% --> _numero Il numero dopo l'underscore lo rende un pedice 
% --> ^numero Il cappellino trasforma il numero in un apice
% \times --> Aggiunge il simbolo della moltiplicazione
% \sqrt[numero] --> Mette tutto sotto una radice quadrata (se non c'è la prentesi quadra), se voglio fare una radice cubica devo scrivere il 3 all'interno delle parentesi quadre
% \sum P_i --> Simbolo della sommatoria per la variabile P_i

% \label{eq:prosciuttomelone} --> Crea una label/etichetta in cui indico a che tipo di oggetto fa riferimento
% \ref{label} --> Va a richiamare la "label" nel testo e se uso il pacchetto hyperref crea anche un link direttamente alla funzione








% [] for structures!

\documentclass[a4paper, 12pt]{article}
\usepackage{graphicx} % Required for inserting images
\usepackage{lineno}
\usepackage{hyperref}
\usepackage{natbib}
\usepackage{color}
\linenumbers
\newcommand{\tb}{\textcolor{blue}}
\linespread{1.5}


% TeXLive
% kile

\begin{document}

\maketitle

$^1$ UNIBO - via Irnerio....

\tableofcontents

\begin{abstract}
The house stood on a slight rise just on the edge of the village. It stood on its own and looked out over a broad spread of West Country farmland. Not a remarkable house by any means—it was about thirty years old, squattish, squarish, made of brick, and had four windows set in the front of a size and proportion which more or less exactly failed to please the eye.
\end{abstract}

\section{Introduction} % sections with no numbers

\tb{The only person for whom the house was in any way special was Arthur Dent, and that was only because it happened to be the one he lived in.} As we will see in Section \ref{sec:methods}, he had lived in it for about three years, ever since he had moved out of London because it made him nervous and irritable. \tb{He was about thirty as well}, tall, dark-haired and never quite at ease with himself. The thing that used to worry him most was the fact that people always used to ask him what he was looking so worried about. He worked in local radio which he always used to tell his friends was a lot more interesting than they probably thought. It was, too—most of his friends worked in advertising.

The only person for whom the house was in any way special was Arthur Dent, and that was only because it happened to be the one he lived in. He had lived in it for about three years, ever since he had moved out of London because it made him nervous and irritable. He was about thirty as well, tall, dark-haired and never quite at ease with himself. The thing that used to worry him most was the fact that people always used to ask him what he was looking so worried about. He worked in local radio which he always used to tell his friends was a lot more interesting than they probably thought. It was, too—most of his friends worked in advertising.

\section{Methods}\label{sec:methods}
\subsection{Study area} % itemize
The thing that used to worry him most was the fact that people always used to ask him what he was looking so worried about (Figure \ref{fig:brenta}). He worked in local radio which he always used to tell his friends was a lot more interesting than they probably thought. It was, too—most of his friends worked in advertising (Figure \ref{fig:lake}).

\noindent The main characteristics of the study area are:
\begin{itemize}
    \item Elevation: 3000 meters
    \item Precipitations: n mm per year
    \item Meam Annual Temperature: 12 degrees
\end{itemize}

\noindent The main characteristics of the study area are:
\begin{enumerate}
    \item Elevation: 3000 meters
    \item Precipitations: n mm per year
    \item Meam Annual Temperature: 12 degrees
\end{enumerate}

\subsection{Statistical analysis}
The calculation done in this manuscript can be viewed in Equation \ref{eq:newton}.

\begin{equation}
    F = \sqrt[3]{G \times \frac{m_1 \times m_2}{r^2}}
    \label{eq:newton}
\end{equation}


\section{Results}

The only person for whom the house was in any way special was Arthur Dent, and that was only because it happened to be the one he lived in. He had lived in it for about three years, ever since he had moved out of London because it made him nervous and irritable. He was about thirty as well, tall, dark-haired and never quite at ease with himself. The thing that used to worry him most was the fact that people always used to ask him what he was looking so worried about. He worked in local radio which he always used to tell his friends was a lot more interesting than they probably thought. It was, too—most of his friends worked in advertising.


\section{Discussion}
The only person for whom the house was in any way special was Arthur Dent, and that was only because it happened to be the one he lived in. He had lived in it for about three years, ever since he had moved out of London because it made him nervous and irritable. He was about thirty as well, tall, dark-haired and never quite at ease with himself. The thing that used to worry him most was the fact that people always used to ask him what he was looking so worried about. He worked in local radio which he always used to tell his friends was a lot more interesting than they probably thought. It was, too—most of his friends worked in advertising \citep{yoccoz2001}.

\begin{thebibliography}{999}
    \bibitem[Yoccoz et al.(2021)]{yoccoz2001}
    Yoccoz, N. G., Nichols, J. D., \& Boulinier, T. (2001). Monitoring of biological diversity in space and time. Trends in Ecology \& Evolution, 16(8), 446-453.
\end{thebibliography}

\begin{figure}
    \centering
    \includegraphics[width=\textwidth]{dolbr.jpg}
    \caption{This is a sketch of the Brenta mountain.}
    \label{fig:brenta}
\end{figure}

\begin{figure}
    \centering
    \includegraphics[width=\textwidth]{lake.jpg}
    \caption{This is a sketch of the Brenta mountain.}
    \label{fig:lake}
\end{figure}

\end{document}
